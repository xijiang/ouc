\documentclass[serif,aspectratio=169]{beamer}
\usepackage{fontspec,xunicode,xltxtra}
\usepackage{hyperref}
\usepackage{amsmath,amssymb,amsfonts}
\usepackage{graphicx}
\usepackage{siunitx}
\usepackage{wasysym}            % male and female symbols

\usefonttheme{professionalfonts}

\XeTeXlinebreaklocale zh
\XeTeXlinebreakskip = 0pt plus 1pt minus 0.1pt

\setmainfont{FangSong}
\renewcommand{\baselinestretch}{1.25}

\AtBeginSection[]
{
  \begin{frame}
    \frametitle{目录}  %\insertsection to display section title                                                                             
    \tableofcontents[currentsection]
  \end{frame}
}

\AtBeginSubsection[]
{
  \begin{frame}
    \frametitle{目录}  %\insertsection to display section title                                                                             
    \tableofcontents[currentsubsection]
  \end{frame}
}

\addtobeamertemplate{navigation symbols}{}{%
  \usebeamerfont{footline}%
  \usebeamercolor[fg]{footline}%
  \hspace{1em}%
  \insertframenumber/\inserttotalframenumber
}
%\setbeamercovered{invisible}

\begin{document}

\title{\fontspec{LiSu}第三讲: A 及其逆矩阵、模型举例,基因组选择}
\author{于希江}
\date{\tiny{~{二〇一九·十月} \\青岛}}
\institute{挪威生命科学大学畜牧与水产系}
\titlegraphic{\includegraphics[height=.1\textwidth]{img/nmbu.png}}


\begin{frame}
  \frametitle{第二天回顾}
  \begin{itemize}
  \item 微效等位基因向后代的传递
  \item 方差、协方差
  \item 多元正态分布
  \item 广义最小二乘的原理
  \item BLUE 是固定效应的广义最小二乘估计
  \item BLUP 和混合线性方程组
  \end{itemize}
\end{frame}


\begin{frame}
  \frametitle{两种协方差}
  \begingroup
  \fontsize{8pt}{8pt}\selectfont
  \begin{columns}
    \begin{column}{.5\textwidth}
      \begin{block}{单亲与其一子}
        \begin{itemize}
        \item 模型
          \begin{description}
          \item [亲] $y_i=\mu+a_i+e_i$
          \item [子] $y_o=\mu+a_o+e_o'=\mu+\frac{a_i}{2}+e_o$
          \end{description}
        \item 协方差
          \begin{align*}
            \mathrm{Cov}(y_i, y_o)&=\mathrm{E}(\mu+a_i+e_i-\mu)(\mu+\frac{a_i}{2}+e_o-\mu)\\
            &=\mathrm{E}(a_i+e_i)(\frac{a_i}{2}+e_o)\\
            &=\mathrm{E}(a_i\cdot\frac{a_i}{2}+\overbrace{e_i\cdot\frac{a_i}{2}}^0+\overbrace{a_i\cdot e_o}^0+\overbrace{e_i\cdot e_o}^0)\\
            &=\mathrm{E}\left(\frac{a_i^2}{2}\right) = \frac{\sigma_a^2}{2}
          \end{align*}
        \end{itemize}
      \end{block}
    \end{column}
}


\frame{
  \titlepage
}


\section{A 及其逆矩阵}
\subsection{A 矩阵}
\begin{frame}
  \frametitle{两个半同胞个体}
  \begin{align*}
    \mathrm{Var}(\mathbf{y}) &= \mathbf{ZGZ'}+\mathbf{R}\\
    \mathbf{Z}&=\left[\begin{array}{cc}1&0\\0&1\end{array}\right]=\mathbf{I}\\
    \mathbf{R}&=\mathbf{I}\sigma_e^2\\
    \mathbf{G}&=\left[\begin{array}{cc}1&.25\\.25&1\end{array}\right]\sigma_a^2\\
    &={\color{red}\mathbf{A}}\sigma_a^2
  \end{align*}
\end{frame}


\begin{frame}
  \frametitle{IBD 和 IBS}
  \begin{description}
  \item [IBD] Identical by descent. 指两个基因是同一个基因的复制品。
  \item [IBS] Identical by state. 指两个基因(序列)相同。
  \end{description}
  \begin{itemize}
  \item 通常用一个个体基因组中 IBD 的比例来表示个体的近交程度。
  \item 通常用两个个体之间的 IBD 比例来表示这两个个体之间的亲缘关系。
  \end{itemize}
\end{frame}


\begin{frame}
  \frametitle{加性关系矩阵}
  \begin{columns}
    \begin{column}{.5\textwidth}
      \begin{itemize}
      \item $\mathbf{A}$ 可以用表格法递推
      \item $\mathbf{A}^{-1}$ 也可以直接构建。
        \begin{itemize}
        \item Henderson 1975
        \item Meuwissen and Luo 1992
        \end{itemize}
      \item $\mathbf{A}$ 的元素相当于 Wright 遗传相关的分子
        \begin{itemize}
        \item 因此又称分子血缘相关系数矩阵。
        \end{itemize}
      \end{itemize}
    \end{column}

    \pause
    \begin{column}{.5\textwidth}
      \begin{table}
        \caption{一个排序后的系谱示例}
        \begin{tabular}{ccc}
          个体 & 父 & 母\\\hline
          A & - & -\\
          B & - & -\\
          C & - & -\\
          D & A & B\\
          E & A & C\\
          F & E & D
        \end{tabular}
      \end{table}
    \end{column}
  \end{columns}
\end{frame}


\begin{frame}
  \frametitle{表格法}
  \begin{columns}
    \begin{column}{.5\textwidth}
      \centering
      \begin{tabular}{c|ccc|ccc}
        & -,- & -,- & -,- & A,B & A,C & E,D\\
        & A & B & C & D & E & F\\\hline
        A & 1 & 0 & 0\\
        B & 0 & 1 & 0\\
        C & 0 & 0 & 1\\\hline
        D & & &\\
        E & & &\\
        F & & &
      \end{tabular}
    \end{column}

    \begin{column}{.5\textwidth}
      \begin{itemize}
      \item 父母未知的个体
        \begin{itemize}
        \item 令其对角线为元素 1
        \item 互相之间的关系为 0
        \end{itemize}
      \end{itemize}
    \end{column}
  \end{columns}
\end{frame}


\begin{frame}
  \frametitle{表格法\tiny{续一}}
  \begin{columns}
    \begin{column}{.5\textwidth}
      \centering
      \begin{tabular}{c|ccc|ccc}
        & -,- & -,- & -,- & A,B & A,C & E,D\\
        & A & B & C & D & E & F\\\hline
        A & 1 & 0 & 0 & .5 & .5 & .5\\
        B & 0 & 1 & 0 & .5 & 0 & .25\\
        C & 0 & 0 & 1 & 0 & .5 & .25\\\hline
        D & .5 & .5 & 0\\
        E & .5 & 0 & .5\\
        F & .5 & .25 & .25
      \end{tabular}
    \end{column}

    \begin{column}{.5\textwidth}
      \begin{align*}
        a_{\mathrm{AD}} = .5(a_{\mathrm{AA}}+a_{\mathrm{AB}}) &= .5(1+0)=.5\\
        a_{\mathrm{AE}} = .5(a_{\mathrm{AA}}+a_{\mathrm{AC}}) &= .5(1+0)=.5\\
        a_{\mathrm{AF}} = .5(a_{\mathrm{AE}}+a_{\mathrm{AD}}) &= .5(.5+.5)=.5
      \end{align*}
      一般的,若 $j$ 的父母为 $p$ 和 $q$,则:
      $$\color{red}a_{ij} = .5(a_{ip} + a_{iq})$$
    \end{column}
  \end{columns}
\end{frame}


\begin{frame}
  \frametitle{表格法\tiny{续二}}
  \begin{columns}
    \begin{column}{.5\textwidth}
      \centering
      \begin{tabular}{c|ccc|ccc}
        & -,- & -,- & -,- & A,B & A,C & E,D\\
        & A & B & C & D & E & F\\\hline
        A & 1 & 0 & 0 & .5 & .5 & .5\\
        B & 0 & 1 & 0 & .5 & 0 & .25\\
        C & 0 & 0 & 1 & 0 & .5 & .25\\\hline
        D & .5 & .5 & 0 & 1 & .25 & .625\\
        E & .5 & 0 & .5 & .25 & 1 & .625\\
        F & .5 & .25 & .25 & .625 & .625 & 1.125
      \end{tabular}
    \end{column}

    \begin{column}{.5\textwidth}
      父母已知的个体,依定义,其近交系数 $F_i$ 为其父母的加性遗传关系的一半。如:
      $$F_{\mathrm{D}} = .5a_{\mathrm{AB}} = 0$$

      个体 D 与其本身的加性遗传关系为:
      $$a_{\mathrm{DD}} = 1+F_{\mathrm{D}} = 1+0$$
      一般的,若个体 $i$ 的父母为 $p$ 和 $q$:
      $$\color{red}a_{ii} = 1+.5a_{pq}$$
    \end{column}
  \end{columns}
\end{frame}


\subsection{A 的逆矩阵}
\begin{frame}
  \frametitle{一般规则}
  \begin{itemize}
  \item 当 $\mathbf{A}$ 维数小时,可以直接计算 $\mathbf{A}^{-1}$
    \begin{itemize}
    \item 如果是根据(SNP)基因型计算的 $\mathbf{G}$,则只能硬算 $\mathbf{G}^{-1}$
    \item 或者用特殊的数值算法
    \end{itemize}
  \item 根据系谱,有简单的方法直接计算 $\mathbf{A}^{-1}$,而不用计算$\mathbf{A}$
    \begin{itemize}
    \item Henderson 1976;Quaas, 1976
    \end{itemize}
  \item 见 {\color{cyan}day-3-a-mat.ipynb}
  \end{itemize}
\end{frame}


\section{模型示例}
\begin{frame}
  \frametitle{公牛模型}
  \begin{columns}
    \begin{column}{.4\textwidth}
      $$\mathbf{y=xb+zu+e}$$
      \begin{itemize}
      \item 有来自 2 个奶牛场的 2 头公牛的 5 头女儿的记录如下:
      \end{itemize}
      \centering
      \begin{tabular}{rrrr}
        个体号 & 场 & 公牛 & 表型值\\\hline
        1 & 1 & 1 & 11\\
        2 & 1 & 1 & 15\\
        3 & 2 & 1 & 10\\
        4 & 1 & 2 & 19\\
        5 & 2 & 2 & 25
      \end{tabular}
    \end{column}
    
    \pause
    \begin{column}{.6\textwidth}
      $$
      \color{cyan}\left[\begin{array}{ll}
          \mathbf{X}'\mathbf{X} & \mathbf{X}'\mathbf{Z}\\
          \mathbf{Z}'\mathbf{X} & \mathbf{Z}'\mathbf{Z}+\mathbf{A}^{-1}\lambda
        \end{array}\right]
      \left[\begin{array}{cc}
          \hat{\mathbf{b}}\\
          \hat{\mathbf{u}}
        \end{array}\right] = 
      \left[\begin{array}{c}
          \mathbf{X}'\mathbf{y}\\
          \mathbf{Z}'\mathbf{y}
        \end{array}\right]
      $$
    \end{column}
  \end{columns}
\end{frame}


\begin{frame}
  \frametitle{线性混合模型观测值举例}
  \begin{itemize}
  \item 公牛模型
  \end{itemize}
  $$
  \begin{array}{ccccccccc}
    \mathbf{y} & = & \mathbf{y} & = & \mathbf{Xb} & + & \mathbf{Zu} & + & \mathbf{e}\\
    \left[\begin{array}{c}
        y_{111}\\
        y_{112}\\
        y_{213}\\
        y_{121}\\
        y_{222}
      \end{array}\right] & = &
    \left[\begin{array}{c}
        11\\
        15\\
        10\\
        19\\
        25
      \end{array}\right] & = &
    \left[\begin{array}{cc}
        1 & 0\\
        1 & 0\\
        0 & 1\\
        1 & 0\\
        0 & 1
      \end{array}\right]
    \left[\begin{array}{c}b_1\\b_2\end{array}\right] & + &
    \left[\begin{array}{cc}
        1 & 0\\
        1 & 0\\
        1 & 0\\
        0 & 1\\
        0 & 1
      \end{array}\right]
    \left[\begin{array}{c}u_1\\u_2\end{array}\right] & + &
    \left[\begin{array}{c}
        e_{111}\\
        e_{112}\\
        e_{213}\\
        e_{121}\\
        e_{222}
      \end{array}\right]
  \end{array}
  $$
  
\end{frame}

\begin{frame}
  \frametitle{混合线性模型方程组}
  \begin{columns}
    \begin{column}{.4\textwidth}
      令 $h^2=.25$
      \begin{align*}
        \lambda &= \frac{\sigma_e^2}{\sigma_u^2}\\
        &=\frac{(1-.25h^2)\sigma_P^2}{.25h^2\sigma_P^2}\\
        &=\frac{1-.25h^2}{.25h^2}\\
        &=\frac{4-h^2}{h^2}\\
        &=\frac{4-.25}{.25}=15
      \end{align*}
    \end{column}

    \begin{column}{.6\textwidth}
      若两头公牛无亲缘关系,则 $\mathbf{A=I=A}^{-1}$:
      
      $$
      \left[\begin{array}{cccc}
          3 & 0 & 2 & 1\\
          0 & 2 & 1 & 1\\
          2 & 1 & 3+\lambda & 0\\
          1 & 1 & 0 & 2+\lambda
        \end{array}\right]
      \left[\begin{array}{c}
          \hat{b}_1\\
          \hat{b}_2\\
          \hat{u}_1\\
          \hat{u}_2
        \end{array}\right]=
      \left[\begin{array}{c}
          45\\
          35\\
          36\\
          44
        \end{array}\right]
      $$
      
    \end{column}
  \end{columns}
\end{frame}


\begin{frame}
  \frametitle{MME}
  \begin{columns}
    \begin{column}{.6\textwidth}
      $$
      \left[\begin{array}{rrrr}
          3 & 0 & 2 & 1\\
          0 & 2 & 1 & 1\\
          2 & 1 & 18 & 0\\
          1 & 1 & 0 & 17
        \end{array}\right]
      \left[\begin{array}{c}
          \hat{b}_1\\
          \hat{b}_2\\
          \hat{u}_1\\
          \hat{u}_2
        \end{array}\right]=
      \left[\begin{array}{c}
          45\\
          35\\
          36\\
          44
        \end{array}\right]
      $$

    \end{column}
    \pause
    \begin{column}{.4\textwidth}
      
      $$
      \left[\begin{array}{c}
          \hat{b}_1\\
          \hat{b}_2\\
          \hat{u}_1\\
          \hat{u}_2
        \end{array}\right]=
      \left[\begin{array}{r}
          15.22\\
          17.50\\
          -0.66\\
          0.66
        \end{array}\right]
      $$
      
    \end{column}
  \end{columns}
\end{frame}


\begin{frame}
  \frametitle{动物模型}
  \begin{columns}
    \begin{column}{.5\textwidth}
      \begin{block}{模型}
        $$y_i = \mu+a_i+e_i$$
      \end{block}
      \begin{block}{假设}
        \begin{enumerate}
        \item 群体足够大
        \item 后代无选择
        \item 假定只有加性作用
        \item 除了模型中的效应,无其它作用
        \end{enumerate}
      \end{block}
    \end{column}
    
    \begin{column}{.5\textwidth}
      \begin{align*}
        \mathrm{E}(a_i) &= 0,\qquad \mathrm{E}(e_i)=0\\
        \mathrm{Var}(a_i) &=\sigma_a^2\\
        \mathrm{Var}(e_i) &=\sigma_e^2\\
        \mathrm{Var}(\mathbf{a}) &= \mathbf{A}\sigma_a^2\\
        h^2 &=\frac{\sigma_a^2}{\sigma_a^2+\sigma_e^2}\\
        \lambda &=\frac{\sigma_e^2}{\sigma_a^2}
      \end{align*}
    \end{column}
  \end{columns}
\end{frame}


\begin{frame}
  \frametitle{示例}
  \begin{columns}
    \begin{column}{.8\textwidth}
      \centering
      \begin{tabular}{cccc|cccc}
        ID & Pa & Ma & PT & ID & Pa & Ma & PT \\\hline
        1 & 0 & 0 &  -   &  9 & 5 & 6 & 36.0\\
        2 & 0 & 0 &  -   & 10 & 7 & 8 & 66.4\\
        3 & 0 & 0 &  -   & 11 & 5 & 8 & 28.9\\
        4 & 0 & 0 &  -   & 12 & 7 & 6 & 73.0\\
        5 & 1 & 2 & 38.5 & 13 & 1 & 6 & 44.2\\
        6 & 3 & 4 & 48.9 & 14 & 3 & 8 & 53.4\\
        7 & 3 & 2 & 64.3 & 15 & 5 & 4 & 33.6\\
        8 & 1 & 4 & 50.5 & 16 & 7 & 8 & 49.5 
      \end{tabular}
    \end{column}

    \begin{column}{.2\textwidth}
      \begin{align*}
        \sigma_a^2&=36\\
        \sigma_e^2&=64\\
        h^2 &=.36\\
        \lambda&=\frac{16}{9}
      \end{align*}
    \end{column}
  \end{columns}
\end{frame}


\begin{frame}
  \frametitle{每个表型观测值的模型}
  \begingroup
  \fontsize{7pt}{7pt}\selectfont
  $$
  \left[\begin{array}{c}
      1\\1\\1\\1\\1\\1\\1\\1\\1\\1\\1\\1
    \end{array}\right]\mu+
  \left[\begin{array}{cccccccccccccccc}
      0 & 0 & 0 & 0 & {\color{cyan}1} & 0 & 0 & 0 & 0 & 0 & 0 & 0 & 0 & 0 & 0 & 0\\
      0 & 0 & 0 & 0 & 0 & {\color{cyan}1} & 0 & 0 & 0 & 0 & 0 & 0 & 0 & 0 & 0 & 0\\
      0 & 0 & 0 & 0 & 0 & 0 & {\color{cyan}1} & 0 & 0 & 0 & 0 & 0 & 0 & 0 & 0 & 0\\
      0 & 0 & 0 & 0 & 0 & 0 & 0 & {\color{cyan}1} & 0 & 0 & 0 & 0 & 0 & 0 & 0 & 0\\
      0 & 0 & 0 & 0 & 0 & 0 & 0 & 0 & {\color{cyan}1} & 0 & 0 & 0 & 0 & 0 & 0 & 0\\
      0 & 0 & 0 & 0 & 0 & 0 & 0 & 0 & 0 & {\color{cyan}1} & 0 & 0 & 0 & 0 & 0 & 0\\
      0 & 0 & 0 & 0 & 0 & 0 & 0 & 0 & 0 & 0 & {\color{cyan}1} & 0 & 0 & 0 & 0 & 0\\
      0 & 0 & 0 & 0 & 0 & 0 & 0 & 0 & 0 & 0 & 0 & {\color{cyan}1} & 0 & 0 & 0 & 0\\
      0 & 0 & 0 & 0 & 0 & 0 & 0 & 0 & 0 & 0 & 0 & 0 & {\color{cyan}1} & 0 & 0 & 0\\
      0 & 0 & 0 & 0 & 0 & 0 & 0 & 0 & 0 & 0 & 0 & 0 & 0 & {\color{cyan}1} & 0 & 0\\
      0 & 0 & 0 & 0 & 0 & 0 & 0 & 0 & 0 & 0 & 0 & 0 & 0 & 0 & {\color{cyan}1} & 0\\
      0 & 0 & 0 & 0 & 0 & 0 & 0 & 0 & 0 & 0 & 0 & 0 & 0 & 0 & 0 & {\color{cyan}1}
    \end{array}\right]
  \left[\begin{array}{l}
      u_1 \\
      u_2 \\
      u_3 \\
      u_4 \\
      u_5 \\
      u_6 \\
      u_7 \\
      u_8 \\
      u_9 \\
      u_{10}\\
      u_{11}\\
      u_{12}\\
      u_{13}\\
      u_{14}\\
      u_{15}\\
      u_{16}
    \end{array}\right]+
  \left[\begin{array}{c}
      e_5\\
      e_6\\
      e_7\\
      e_8\\
      e_9\\
      e_{10}\\
      e_{11}\\
      e_{12}\\
      e_{13}\\
      e_{14}\\
      e_{15}\\
      e_{16}
    \end{array}\right]=
  \left[\begin{array}{c}
      38.5\\
      48.9\\
      64.3\\
      50.5\\
      36.0\\
      66.4\\
      28.9\\
      73.0\\
      44.2\\
      53.4\\
      33.6\\
      49.5
    \end{array}\right]
  $$
  \endgroup
\end{frame}


\begin{frame}
  \begingroup
  \fontsize{8pt}{8pt}\selectfont
  $$
  \mathbf{A}=\frac{1}{16}\left[\begin{array}{rrrrrrrrrrrrrrrr}
      16\\
      0 & 16 &  &  &  & & s & y & m & m & e & t & r & y\\
      0 & 0 & 16\\
      0 & 0 & 0 & 16\\
      8 & 8 & 0 & 0 & 16\\
      0 & 0 & 8 & 8 & 0 & 16\\
      0 & 8 & 8 & 0 & 4 & 4 & 16\\
      8 & 0 & 0 & 8 & 4 & 4 & 0 & 16\\
      4 & 4 & 4 & 4 & 8 & 8 & 4 & 4 & 16\\
      4 & 4 & 4 & 4 & 4 & 4 & 8 & 8 & 4 & 16\\
      8 & 4 & 0 & 4 & 10 & 2 & 2 & 10 & 6 & 6 & 18\\
      0 & 4 & 8 & 4 & 2 & 10 & 10 & 2 & 6 & 6 & 2 & 18\\
      8 & 0 & 4 & 4 & 4 & 8 & 2 & 6 & 6 & 4 & 5 & 5 & 16\\
      4 & 0 & 8 & 4 & 2 & 6 & 4 & 8 & 4 & 6 & 5 & 5 & 5 & 16\\
      4 & 4 & 0 & 8 & 8 & 4 & 2 & 6 & 6 & 4 & 7 & 3 & 4 & 3 & 16\\
      4 & 4 & 4 & 4 & 4 & 4 & 8 & 8 & 4 & 8 & 6 & 6 & 4 & 6 & 4 & 16
    \end{array}\right]
  $$
  $\mathbf{X,\,Z,\,y}$ 已知,$\mathbf{G=A}_{16}\sigma_a^2$,$\mathbf{R=I}_{12}\sigma_e^2$
  \endgroup
\end{frame}


\begin{frame}
  \frametitle{混合线性模型方程组}
  $$
  \color{cyan}\left[\begin{array}{ll}
      \mathbf{X}'\mathbf{X} & \mathbf{X}'\mathbf{Z}\\
      \mathbf{Z}'\mathbf{X} & \mathbf{Z}'\mathbf{Z}+\mathbf{A}^{-1}\lambda
    \end{array}\right]
  \left[\begin{array}{cc}
      \hat{\mathbf{b}}\\
      \hat{\mathbf{u}}
    \end{array}\right] = 
  \left[\begin{array}{c}
      \mathbf{X}'\mathbf{y}\\
      \mathbf{Z}'\mathbf{y}
    \end{array}\right]
  $$
\end{frame}

\section{基因组选择简介}
\begin{frame}
  \begin{columns}
    \begin{column}{.5\textwidth}
      \begin{block}{若干概念}
        \begin{description}
        \item [EBV] 传统的估计育种值,如来自 BLUP 的。
        \item [DGV] Direct genmoic BV.
        \item [GEBV] Genomically enhanced/enabled/estimated BV
        \item [SNP] Single nucleotype polymorphism
        \item [Chips] 芯片
        \end{description}
      \end{block}
    \end{column}

    \pause
    \begin{column}{.5\textwidth}
      \begin{block}{一个著名结果}
        \begin{table}
          \caption{Meuwissen et al., 2001}
          \begin{tabular}{rcc}
            & $r_{\mathrm{TBV,EBV}}$ & $b_{\mathrm{TBV,EBV}}$\\\hline
            LS & 0.32 & 0.29\\
            genomic BLUP & 0.73 & 0.90\\
            Bayes A & 0.80 & 0.83\\
            Bayes B & 0.85 & 0.95
          \end{tabular}
        \end{table}
      \end{block}
    \end{column}
  \end{columns}
\end{frame}


\begin{frame}
  \frametitle{基因型信息}
  \begin{block}{基因组选择之前}
    \begin{itemize}
    \item 用于淘汰或固定某些基因,例如遗传缺陷、疾病
    \item 有一些标记被用于标记辅助选择,以改进育种值的估计
    \end{itemize}
  \end{block}

  \pause
  \begin{block}{基因组选择}
    \begin{itemize}
    \item 用到成千上万的基因座位
    \item 对单个基因的作用通常不再感兴趣
    \item 目的要完全通过这些标记基因(与 QTL 的 LD)来估计育种值
    \end{itemize}
  \end{block}
\end{frame}


\begin{frame}
  \frametitle{一般模型}
  $$\mathbf{y=1_n\mu+Zg+e}$$
  \begin{itemize}
  \item 其中
    \begin{description}
    \item [$\mathbf{y}$] $n\times 1$ 观察值向量
    \item [$\mathbf{1}_n$] 元素为 1 的向量
    \item [$\mu$] 未知的一般平均数
    \item [$\mathbf{Z}$] $n\times m$ 的标记基因型矩阵
    \item [$\mathbf{g}$] $m\times 1$ (随机的)SNP 标记效应
    \item [$\mathbf{e}$] 随机残差向量
    \end{description}
  \end{itemize}
\end{frame}


\begin{frame}
  \frametitle{基因组选择}
  \begin{itemize}
  \item 需要的数据
    \begin{itemize}
    \item 基因型
    \item 表现型
    \item 系谱【可选】
    \end{itemize}
  \item 关于 $\mathbf{Z}$
    \begin{itemize}
    \item 筛除低质量的标记
    \item 有很多基因型缺失的座位(如:$>20$\%)
    \item 设置 MAF,如 1\%
    \item 缺失基因型的填充(imputation)
    \item 根据系谱检查基因型的正确性。
    \item 去除有争议的基因型
    \end{itemize}
  \end{itemize}
\end{frame}


\section{SNP-BLUP}
\begin{frame}
  \frametitle{SNP-BLUP}
  \begin{columns}
    \begin{column}{.5\textwidth}
      $$\mathbf{y=1_n\mu+Zg+e}$$
      \begin{itemize}
      \item 其中
        \begin{description}
        \item [$\mathbf{y}$] $n\times 1$ 观察值向量
        \item [$\mathbf{1}_n$] 元素为 1 的向量
        \item [$\mu$] 未知的一般平均数
        \item [$\mathbf{Z}$] $n\times m$ 的标记基因型矩阵
        \item [$\mathbf{g}$] $m\times 1$ (随机的)SNP 标记效应
        \item [$\mathbf{e}$] 随机残差向量
        \end{description}
      \end{itemize}
    \end{column}

    \begin{column}{.5\textwidth}
      \begin{itemize}
      \item 假定:
        \begin{description}
        \item [$\mathbf{e}\sim\mathrm{MVN}(\mathbf{0}, \mathbf{R})$] 其中 $\mathbf{R=R}_0\sigma_e^2$,且残差方差 $\sigma_e^2$ 已知。
        \item [$\mathbf{g}\sim\mathrm{MVN}(\mathbf{0}, \mathbf{I}\sigma_g^2)$] 有已知的 $\sigma_g^2$。
        \end{description}
      \end{itemize}
    \end{column}
  \end{columns}
\end{frame}


\begin{frame}
  \frametitle{SNP-BLUP 混合模型方程组}
  \begin{columns}
    \begin{column}{.6\textwidth}
      \begin{block}{估计标记基因型效应的 MME}
        $$
        \left[\begin{array}{ll}
            \mathbf{1}'\mathbf{R}^{-1}\mathbf{1} & \mathbf{1}'\mathbf{R}^{-1}\mathbf{Z}\\
            \mathbf{Z}'\mathbf{R}^{-1}\mathbf{1} & \mathbf{Z}'\mathbf{R}^{-1}\mathbf{Z}+\mathbf{I}^{-1}/\sigma_g^2
          \end{array}\right]
        \left[\begin{array}{cc}
            \hat{\mathbf{\mu}}\\
            \hat{\mathbf{g}}
          \end{array}\right] = 
        \left[\begin{array}{c}
            \mathbf{1}'\mathbf{R}^{-1}\mathbf{y}\\
            \mathbf{Z}'\mathbf{R}^{-1}\mathbf{y}
          \end{array}\right]
        $$
      \end{block}

      \begin{block}{当 $\mathbf{R=I}\sigma_e^2$, $\lambda=\frac{\sigma_e^2}{\sigma_g^2}$}
        $$
        \left[\begin{array}{ll}
            \mathbf{1}'\mathbf{1} & \mathbf{1}'\mathbf{Z}\\
            \mathbf{Z}'\mathbf{1} & \mathbf{Z}'\mathbf{Z}+\mathbf{I}\lambda
          \end{array}\right]
        \left[\begin{array}{cc}
            \hat{\mathbf{b}}\\
            \hat{\mathbf{u}}
          \end{array}\right] = 
        \left[\begin{array}{c}
            \mathbf{X}'\mathbf{y}\\
            \mathbf{Z}'\mathbf{y}
          \end{array}\right]
        $$
      \end{block}
    \end{column}

    \pause
    \begin{column}{.4\textwidth}
      \begin{itemize}
      \item 基因组育种值
        \begin{itemize}
        \item 训练组(training set)
          $$\hat{\mathbf{u}}_t = \mathbf{1}\hat{\mu}+\mathbf{Z}_t\hat{\mathbf{g}}$$
        \item 候选组(candidate set)
          $$\hat{\mathbf{u}}_c = \mathbf{1}\hat{\mu}+\mathbf{Z}_c\hat{\mathbf{g}}$$
        \end{itemize}
      \end{itemize}
    \end{column}
  \end{columns}
\end{frame}


\section{G-BLUP}
\begin{frame}
  \frametitle{G-BLUP}
  \begin{columns}
    \begin{column}{.5\textwidth}
      \begin{block}{G-BLUP}
        \begin{itemize}
        \item 直接估计育种值,而不估计标记效应
        \item 使用基因组(关系)矩阵
        \item 得到与 SNP-BLUP 相同的结果
        \end{itemize}
      \end{block}
    \end{column}

    \pause
    \begin{column}{.5\textwidth}
      \begin{block}{由 $\mathbf{a=Zg}$}
        \begin{align*}
          \mathbf{y} &=\mathbf{1\mu+a+e}\\
          &\Rightarrow\mathrm{Var}(\mathbf{a})=\mathrm{Var}(\mathbf{Zg})=\mathbf{Z}\mathrm{Var}(\mathbf{g})\mathbf{Z}'\\
          \textrm{if } & \mathrm{Var}(\mathbf{g}) =\mathbf{I}\sigma_g^2\\
          & \Rightarrow\mathbf{a}\sim\mathrm{MVN}(\mathbf{0,G}\sigma_g^2)\\
          \textrm{where } &\mathbf{G=ZZ}'
        \end{align*}
      \end{block}
    \end{column}
  \end{columns}
\end{frame}


\begin{frame}
  \frametitle{所有个体的 G-BLUP}
  与动物模型相似,候选个体有基因组信息,但是没有观测结果。

  我们假定:

  $$\mathbf{y=1\mu+Wa+e}$$

  $\mathbf{W}$ 是 $n\times q$的指示矩阵。

  $q=n_t+n_c$。其中 $n_t$ 是有观测值的训练组。$n_c$ 是有基因型无表型观测值的候选群体。

  $\mathbf{a=Zg}$, $\mathbf{Z}=\left[\begin{array}{c}\mathbf{Z}_t\\\mathbf{Z}_c\end{array}\right]$, $\mathbf{a} = \left[\begin{array}{c}\mathbf{a}_t\\\mathbf{a}_c\end{array}\right]$。  \pause$\mathbf{a}\sim\mathrm{MVN}(\mathbf{0,G}\sigma_g^2)$, $\mathbf{G=ZZ}'$
\end{frame}


\begin{frame}
  \frametitle{MME, 仅训练组}
  \begin{block}{MME}
    $$
    \left[\begin{array}{ll}
        \mathbf{1}'_n\mathbf{R}^{-1}\mathbf{1}_n & \mathbf{1}'_n\mathbf{R}^{-1}\\
        \mathbf{R}^{-1}\mathbf{1}_n & \mathbf{R}^{-1}+\mathbf{G}^{-1}_t/\sigma_g^2
      \end{array}\right]
    \left[\begin{array}{c}
        \hat{\mu}\\\hat{\mathbf{a}}_t
      \end{array}\right]=
    \left[\begin{array}{c}
        \mathbf{1}'_n\mathbf{R}^{-1}\mathbf{y}\\\mathbf{R}^{-1}\mathbf{y}
      \end{array}\right]
    $$
  \end{block}
  
  \begin{block}{当 $\mathbf{R=I}\sigma_e^2$}
    $$
    \left[\begin{array}{ll}
        \mathbf{1}'_n\mathbf{1}_n & \mathbf{1}'_n\\
        \mathbf{1}_n & \mathbf{I}_n+\lambda\mathbf{G}^{-1}_t
      \end{array}\right]
    \left[\begin{array}{c}
        \hat{\mu}\\\hat{\mathbf{a}}_t
      \end{array}\right]=
    \left[\begin{array}{c}
        \mathbf{1}'_n\mathbf{y}\\\mathbf{y}
      \end{array}\right]
    $$

    $\lambda=\frac{\sigma_e^2}{\sigma_g^2}$
  \end{block}
\end{frame}


\begin{frame}
  \frametitle{MME, 所有个体}
  \begin{block}{MME}
    $$
    \left[\begin{array}{ll}
        \mathbf{1}'_n\mathbf{R}^{-1}\mathbf{1}_n & \mathbf{1}'_n\mathbf{R}^{-1}\mathbf{W}\\
        \mathbf{W}'\mathbf{R}^{-1}\mathbf{1}_n & \mathbf{W}'\mathbf{R}^{-1}\mathbf{W}+\mathbf{G}^{-1}_t/\sigma_g^2
      \end{array}\right]
    \left[\begin{array}{c}
        \hat{\mu}\\\hat{\mathbf{a}}_t
      \end{array}\right]=
    \left[\begin{array}{c}
        \mathbf{1}'_n\mathbf{R}^{-1}\mathbf{y}\\\mathbf{W}'\mathbf{R}^{-1}\mathbf{y}
      \end{array}\right]
    $$
  \end{block}
  
  \begin{block}{当 $\mathbf{R=I}\sigma_e^2$}
    $$
    \left[\begin{array}{ll}
        \mathbf{1}'_n\mathbf{1}_n & \mathbf{1}'_n\mathbf{W}\\
        \mathbf{W}'\mathbf{1}_n & \mathbf{W'W}+\lambda\mathbf{G}^{-1}_t
      \end{array}\right]
    \left[\begin{array}{c}
        \hat{\mu}\\\hat{\mathbf{a}}_t
      \end{array}\right]=
    \left[\begin{array}{c}
        \mathbf{1}'_n\mathbf{y}\\\mathbf{W}'\mathbf{y}
      \end{array}\right]
    $$

    由于 $\mathbf{a}'=[\mathbf{a}'_t\,\mathbf{a}'_c]\Rightarrow\mathbf{W}=[\mathbf{I\,0}]$
  \end{block}
\end{frame}


\begin{frame}
  \frametitle{MME, 所有个体}
  $$
  \left[\begin{array}{ccc}
      \mathbf{1}'_n\mathbf{1}_n & \mathbf{1}'_n & \mathbf{0}\\
      \mathbf{1}_n & \mathbf{I}_n+\lambda\{\mathbf{G}^{-1}\}_{tt} & \lambda\{\mathbf{G}^{-1}\}_{tc}\\
      \mathbf{0} & \lambda\{\mathbf{G}^{-1}\}_{ct} & \lambda\{\mathbf{G}^{-1}\}_{cc}
    \end{array}\right]
  \left[\begin{array}{c}
      \hat{\mu}\\\hat{\mathbf{a}}
    \end{array}\right]=
  \left[\begin{array}{c}
      \mathbf{1}'_n\mathbf{y}\\\mathbf{y}\\\mathbf{0}
    \end{array}\right]
  $$

  其中

  $$
  \mathbf{G}^{-1}=\left[\begin{array}{cc}
      \{\mathbf{G}^{-1}\}_{tt} & \{\mathbf{G}^{-1}\}_{tc}\\
      \{\mathbf{G}^{-1}\}_{ct} & \{\mathbf{G}^{-1}\}_{cc}
    \end{array}\right]
  $$
      
\end{frame}

\end{document}
